% \begin{thebibliography}{3}
%     \bibitem{ya.ru}
%     ya.ru
%     \bibitem{Vshivkov}
%     Григорьев Ю. В., Вшивков В. А., Федорук, М. П. Численное моделирование методами частиц в ячейках. --- Новосибирск : Издательство СО РАН. --- 2004. --- 360 c.
%     \bibitem{LiuLiu}
%     Liu G. R., Liu M. B. Smoothed particle hydrodynamics: a meshfree particle method. --- Singapore : World Scientific Publishing. --- 2003. --- 449 p.
% \end{thebibliography}

% даём указание на включение данного место в оглавление как секции (\section)
\newpage
%далее сам список используевой литературы
\addcontentsline{toc}{section}{Cписок использованных источников}
\bibliographystyle{plain}
\begin{thebibliography}{}
    \bibitem{1} \href{http://www.cs.cornell.edu/courses/cs4758/2012sp/materials/hmm_paper_rabiner.pdf}{www.cs.cornell.edu}
    \bibitem{2} В. И. Левенштейн. Двоичные коды с исправлением выпадений, вставок и
    замещений символов. Доклады Академий Наук СССР, 1965. 163.4:845-848.
    \bibitem{2} Гасфилд. Строки, деревья и последовательности в алгоритмах. Информатика и
    вычислительная биология. Невский Диалект БВХ-Петербург, 2003.
    \bibitem{3} 4.	R. A. Wagner, M. J. Fischer. The string-to-string correction problem. J. ACM 21 1 (1974)
    \bibitem{4} \href{https://books.google.ru/books?id=iEO6AgAAQBAJ&pg=PA8&#v=onepage&q&f=false}{Алгоритмы и процессоры цифровой обработки сигналов.}
    \bibitem{5} \href{https://books.google.ru/books?id=s3H8s8rdsHkC&pg=PA83#v=onepage&q&f=false}{Digital Signal Compression: Principles and Practice.}
    \bibitem{6} \href{https://web.archive.org/web/20090708124031/http://mexman.ru/?p=6}{Теоретическая механика.}
    \bibitem{7} \href{http://journals.ioffe.ru/articles/viewPDF/8381}{Интепритиация измерений оптического гироскопа.}
    \bibitem{8} \href{https://habr.com/ru/post/105220/}{Хабр: Классификация данных методом опорных векторов.}
    \bibitem{9} \href{http://www.cs.cornell.edu/courses/cs4758/2012sp/materials/hmm_paper_rabiner.pdf}{A Tutorial in Hidden Markov Models and Selected Applications in Speech Recognition, LAWRENCE R. REBINER, FELLOW.}
    \bibitem{} \href{https://plotly.com/python-api-reference/generated/plotly.graph_objects.Figure.html}{plotly graph objects Figure}
    \bibitem{} \href{http://www.l3s.de/~anand/tir14/lectures/ws14-tir-foundations-2.pdf}{Periodicity Detection, Time-series Correlation, Burst Detection}
    \bibitem{} \href{https://www.dataq.com/data-acquisition/general-education-tutorials/fft-fast-fourier-transform-waveform-analysis.html}{FFT (Fast Fourier Transform) Waveform Analysis}
    \bibitem{} \href{https://statanaliz.info/statistica/korrelyaciya-i-regressiya/linejnyj-koefficient-korrelyacii-pirsona/}{Линейный коэффициент корреляции Пирсона}
    \bibitem{} \href{http://risktheory.novosyolov.com/topic_correl.htm}{Границы коэффициента корреляци}
    \bibitem{} \href{http://risktheory.novosyolov.com/ill_zerocorr.htm}{Зависимое распределение с нулевой корреляцией}
    \bibitem{} \href{https://s3.amazonaws.com/assets.datacamp.com/production/course_4267/slides/chapter2.pdf}{Autocorrelation Function}
    \bibitem{} \href{https://works.doklad.ru/view/b-Ph-NiLW_k/all.html}{works.doklad.ru}
    \bibitem{} \href{https://orpro.ru/avtokorrelyaciya-primer-resheniya-smotret-stranicy-gde-upominaetsya-termin/}{orpro.ru/avtokorrelyaciya}
    \bibitem{} \href{https://habr.com/ru/post/274175/}{habr.com}
    \bibitem{} \href{http://rachid.koucha.free.fr/tech_corner/pty_pdip.html}{rachid.koucha.free.fr}


\end{thebibliography}