\section{Введение}

Смартфоны -- неотъемлемая часть жизни любого современного человека. С их помощью люди общаются, делают снимки, организуют поездки, делают покупки, следят за новостями. За развитием смартфонов следит буквально весь мир -- из года в год добавляются всё новые функции и возможности. \\

Одной из таких показательных функций является способность смартофона с большой точностью определять угол поворота телефона и вычислять ускорение с помощью встроенных акселерометра и гироскопа. Однако эти приборы не так часто используются -- разве что для ориентации изображения на экране, подсчете шагов в приложениях для спорта и управления в некоторых играх. \\

В рамках данной работы перед командой была поставлена цель -- разработать технологию, позволяющую распознавать различные движения и жесты, совершаемые с помощью мобильного телефона. Для достижения это цели были поставлены глобальные задачи:

\begin{itemize}
    \item Разработать мобильное приложения для сбора данных и тестирования, распознавания и восстановления движений.
    \item Разработать механизм классификации записей отдельного такта движения с акселерометра.
    \item Разработать библиотеку для мобильных устройств, позволяющую в любое приложение интегрировать управление с помощью жестов.
    \item Создать механизм выделения отдельных тактов движения в timeseries акселерометра.
    \item Создать восстановления движения по акселерометру и гироскопу.
    \item Создать механизм сглаживания и фильтрации данных акселерометра и гироскопа
\end{itemize}

\section{Обзор и сравнительный анализ источников по теме проекта \\ (Гусев Владислав)}

Одной из задач являлось исследование уже существующих решений по классификации движений или жестов. Поиск осуществлялся во всех открытых источниках. В свободном доступе готовых решений почти что нет. Поэтому данная задача заняла достаточно большое количество времени.

Считаю нужным отметить, что нас устраивают алгоритмы, которые работают только с аппаратными составляющими – акселерометром и гироскопом. Но, к сожалению, данные, получаемые с этих датчиков, имеют множество шумов, которые создают ряд проблем с обработкой данных с них, но проблема была рассмотрена другим человеком в нашей команде.

Сразу можно разделить алгоритмы по распознаванию жестов на 2 категории: классификация по 3D графику или 3D фигуре, классификация по некому внешнему подобию 3D фигуры (в нашем случае 2D проекции 3D фигуры).
Но методы, использующие 3D модель для идентификации жестов, обрабатывают сложные трехмерные поверхности и классифицируют жесты с помощью нейронных сетей. Следовательно, недостатком такого метода является большая ресурсоемкость, так как построение самой модели, обучение нейронной сети и ее использование могут потребовать значительных ресурсов, что недопустимо при разработке технологии для мобильных устройств. Поэтому было принято решение оставить классификацию движений по 2D проекции.