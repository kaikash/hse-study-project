\section{Введение}

Целью данного проекта является разработка технологии, позволяющей распознавать различные движения и жесты, совершаемые с помощью мобильного телефона. Эта технология должна работать с использованием двух датчиков, встроенных в любой современный смартфон – акселерометра и гироскопа.
Актуальность данной работы заключается в том, что телефоны из года в год включают в себя всё больше и больше функций – наша работа предполагает, что телефон вполне возможно, может использоваться как контроллер и определенные жесты могут использоваться как какие либо команды.
На данный момент подавляющее число приложений распознает только самые просты жесты, вроде встряски при ходьбе или изменения угла поворота в гоночной игре – приложения которое бы распознавало что-то более сложное нет.
Поэтому получаем следующие цели и задачи:

\begin{itemize}
    \item Разработка мобильного приложения для сбора данных и тестирования, распознавания и восстановления движений.
    \item Механизм классификации записей отдельного такта движения с акселерометра.
    \item Механизм восстановления движения по акселерометру и гироскопу.
    \item Библиотека для мобильных устройств, позволяющая в любое приложение интегрировать управление с помощью жестов. 
    \item Создание механизма выделения отдельных тактов движения в timeseries акселерометра.
    \item Создание механизма сглаживания и фильтрации данных акселерометра и гироскопа
\end{itemize}