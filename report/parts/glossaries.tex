\section{Основные термины и определения}

\begin{itemize}
    \item Кажущееся ускорение -- геометрическая разность между действительным (истинным) ускорением объекта и ускорением силы гравитационного притяжения Земли и др. небесных тел.
    \item Акселерометр -- прибор, измеряющий проекцию кажущегося ускорения (разности между истинным ускорением объекта и гравитационным ускорением). Как правило, акселерометр представляет собой чувствительную массу, закреплённую в упругом подвесе. Отклонение массы от её первоначального положения при наличии кажущегося ускорения несёт информацию о величине этого ускорения.
    \item Гироскоп -- устройство, способное реагировать на изменение углов ориентации тела, на котором оно установлено, относительно инерциальной системы отсчёта.
    \item Жест (от лат. gestus — движение тела) -- некоторое действие или движение человеческого тела или его части, имеющее определённое значение или смысл, то есть являющееся знаком или символом.
    \item Магнитометр -- прибор для измерения характеристик магнитного поля и магнитных свойств материалов. В зависимости от измеряемой величины различают приборы для измерения напряжённости поля (эрстедметры), направления поля (инклинаторы и деклинаторы), градиента поля (градиентометры), магнитной индукции (тесламетры), магнитного потока (веберметры, или флюксметры), коэрцитивной силы (коэрцитиметры), магнитной проницаемости (мю-метры), магнитной восприимчивости (каппа-метры), магнитного момента.
    \item Спутниковый приёмник (также GNSS-приёмник) -- радиоприёмное устройство для определения географических координат текущего местоположения антенны приёмника, на основе данных о временных задержках прихода радиосигналов, излучаемых спутниками навигационных систем.
    \item Ускорение свободного падения (ускорение силы тяжести) -- ускорение, придаваемое телу силой тяжести, при исключении из рассмотрения других сил.
    \item Виртуализация на уровне операционной системы (контейнерная виртуализация) — метод виртуализации, при котором ядро операционной системы поддерживает несколько изолированных экземпляров пространства пользователя, вместо одного.
    \item Система управления пакетами — набор программного обеспечения, позволяющего управлять процессом установки, удаления, настройки и обновления различных компонентов программного обеспечения.
    \item Удалённый вызов процедур, реже Вызов удалённых процедур (от англ. Remote Procedure Call, RPC) — класс технологий, позволяющих компьютерным программам вызывать функции или процедуры в другом адресном пространстве (на удалённых компьютерах, либо в независимой сторонней системе на том же устройстве).
    \item Машинное обучение (англ. machine learning, ML) -- класс методов искусственного интеллекта, характерной чертой которых является не прямое решение задачи, а обучение в процессе применения решений множества сходных задач.
    \item Нейронная сеть (также искусственная нейронная сеть, ИНС) -- математическая модель, а также её программное или аппаратное воплощение, построенная по принципу организации и функционирования биологических нейронных сетей — сетей нервных клеток живого организма.
    \item Библиотека (от англ. library) в программировании — сборник подпрограмм или объектов, используемых для разработки программного обеспечения (ПО).
    \item Интерпретируемый язык программирования -- язык программирования, исходный код на котором выполняется методом интерпретации.
    \item Проекция -- отображение точек, фигур, векторов пространства любой размерности на его подпространство любой размерности. (В нашем случае изображение трехмерной фигуры на двумерную плоскость).
    \item Коллинеарность -- отношение параллельности векторов.
    \item Одномерный временной ряд -- собранный в разные моменты времени статистический материал о значениях о каких-либо параметров одного исследуемого процесса.
    \item Многомерный временной ряд -- собранный в разные моменты времени статистический материал о значениях о каких-либо параметров нескольких исследуемых процессов
    \item Быстрое преобразование Фурье -- алгоритм ускоренного вычисления преобразования Фурье, позволяющий получить результат за время, меньшее чем $O(N_2)$.
    \item Спектр данных (функции) -- результат разложения функции (сигнала) на более простые в базисе ортогональных функций.

\end{itemize}